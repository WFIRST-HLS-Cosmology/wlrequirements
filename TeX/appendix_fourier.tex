The shear field on the sky is a spin-2 tensor field, and thus has two components. Such fields can be described using an all-sky formalism \citep{1997PhRvD..55.1830Z, 1997PhRvD..55.7368K}, but since we are interested in small angular scales we use the flat-sky picture here. Gravitational lensing implies a mapping from the source or object plane ${\boldsymbol\theta}^{\rm S}$ to the image (observed) plane ${\boldsymbol\theta}$. Lensing produces a deflection angle ${\boldsymbol\theta}^{\rm S}-{\boldsymbol\theta}\neq{\boldsymbol 0}$, but this is not observable because the ``true'' unlensed position of a galaxy is not known. The lowest-order observable is the gradient of deflection,
\begin{equation}
\frac{\partial \theta^{\rm S}_i}{\partial \theta_j} = \delta_{ij}+A_{ij}, ~~~ {\bf A} = \left(\begin{array}{cc} \kappa+\gamma_+ & \gamma_\times+\omega \\ \gamma_\times-\omega & \kappa-\gamma_+ \end{array} \right).
\end{equation}
The gradient has 4 components: a {\em magnification} $\kappa$; two components of {\em shear} $\gamma_{+,\times}$; and a {\em rotation} $\omega$. Rotation is small (it exists only at second order in the gravitational potential) and is not measurable since the intrinsic orientation of a galaxy is not known; therefore cosmological studies have focused on $\kappa$ (which changes the apparent size and brightness of galaxies) and $\gamma$ (which changes their apparent shape). Since shear studies are more mature than magnification, they have been the primary motivation for WFIRST weak lensing and will be used to set requirements.

The shear components $\gamma_+$ and $\gamma_\times$ as functions of angular position $\theta_x$ and $\theta_y$ decompose as
\begin{equation}
\gamma_+({\boldsymbol\theta},z) = \int \frac{d^2\boldsymbol\ell}{(2\pi)^2} \,\tilde\gamma_+({\boldsymbol\ell},z) \,e^{i{\boldsymbol\ell}\cdot{\boldsymbol\theta}}
~~\leftrightarrow~~
\tilde\gamma_+({\boldsymbol\ell},z) = \int \frac{d^2\boldsymbol\theta}{(2\pi)^2} \,\gamma_+({\boldsymbol\theta},z) \,e^{-i{\boldsymbol\ell}\cdot{\boldsymbol\theta}}.
\end{equation}
In Fourier space, the $E$- and $B$-mode components are defined by rotation of the coordinate system to the basis defined by ${\boldsymbol\ell}$: if $\alpha$ is the position angle of ${\boldsymbol\ell}$, defined by $\cos\alpha = \ell_x/\ell$ and $\sin\alpha = \ell_y/\ell$, then
\begin{equation}
\tilde \gamma_E({\boldsymbol\ell},z) = \cos 2\alpha\,\tilde\gamma_+({\boldsymbol\ell},z) + \sin2\alpha\,\tilde\gamma_\times({\boldsymbol\ell},z)
~~~{\rm and}~~~
\tilde \gamma_B({\boldsymbol\ell},z) = -\sin 2\alpha\,\tilde\gamma_+({\boldsymbol\ell},z) + \cos 2\alpha\,\tilde\gamma_\times({\boldsymbol\ell},z).
\end{equation}
Because weak lensing is generated by a scalar field (the gravitational potential), only the $E$-mode can be generated to first order in the potential. A small $B$-mode can be generated by higher-order effects \citep[e.g.][]{2002A&A...389..729S, 2002ApJ...574...19C, 2010A&A...523A..28K} or by systematic errors.

The shear field has a power spectrum:
\begin{equation}
\langle \tilde\gamma_X^\ast({\boldsymbol\ell},z_i) \tilde\gamma_{X'}({\boldsymbol\ell}',z') \rangle = (2\pi)^2 C_\ell^{XX';z,z'} \delta^{(2)}({\boldsymbol\ell}-{\boldsymbol\ell}').
\end{equation}
Here $X$ and $X'$ are either $E$ or $B$. The $EB$ cross-spectrum is forbidden by parity, and $BB$ is expected to be small, so the $EE$ power spectrum is the primary cosmological channel. If a shear power spectrum is written without specifying the component ($E$ or $B$) then the $EE$ component is implied.

