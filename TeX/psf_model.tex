The PSF models used to assess most requirements are highly simplified. In most cases, the intention of these models, at least in the pre-launch phases, is {\em not} to be accurate to the level of a few parts in $10^4$ (which would require a great deal more sophistication, and has not been achieved by pure first principles modeling in any telescope used for weak lensing -- some empirical tuning of the final PSF model, e.g. by principal components analysis, has always been required). Rather the intention is to address sensitivities: e.g. how far the SM would need to move during an exposure before we care; whether wavefront error jitter is important. As the mission develops, PSF modeling will include many more effects, and we aim in this document to develop requirements that can be flowed to individual sub-groups in the mission (e.g. a STOP analysis should tell us how much of the total PSF error budget is consumed, without having to put in a new model for interpixel capacitance).

We anticipate that models will be developed by several sub-groups in the WFIRST mission. The present models, developers, and principal applications are:
\begin{list}{$\bullet$}{}
\item {\em Model P1: Christopher Hirata (The Ohio State University).} Used for setting requirements on low-order wavefront drift.
\end{list}

\subsubsection{Model P1 (C. Hirata)}

This model used here takes the Fourier transform of an annular pupil with aberrations appearing as contributions to the phase. The resulting ``optical'' PSF is then convolved with a detector response that includes a tophat and charge diffusion. For HgCdTe detectors, we take the charge diffusion length to be 2.94 $\mu$m rms per axis \citep{2007PASP..119..466B}.\footnote{\CMH{That was for an H2RG. We are using this number as a placeholder until we have H4RG data.}} Other effects that are either known to be significant at the level needed for WFIRST weak lensing or are likely to be significant, such as polarized transfer through the telescope, angle of incidence variations in mirror reflectivity, interpixel capacitance in the detectors, etc. are not included. Cases where they may influence stability requirements in a way not captured by the simple pupil model are considered separately.

Model 1 has a default option with the spider turned off (\CMH{implement the latest spider!}). The spider has a major effect on the morphology of the PSF, producing 12 spikes; studies of the AFTA pupil in the past have found effects of the spider on the coefficients herein at the few percent level. The spider further leads to an asymmetric pupil, i.e. with odd-order modes in the decomposition of the amplitude, but this has no appreciable effects on the relation of ellipticity to low-order Zernike modes.\footnote{It is known that an odd-order mode in the phase can mix with other asymmetric phase modes to produce PSF ellipticity, e.g. if one introduces a large trefoil $t$ then the ellipticity develops a linear term in coma, proportional to $tc^\ast$ \citep{2010SPIE.7731E..37N}. However, an amplitude feature with 3-fold or other odd symmetry, such as the spider, does not lead to such an effect.} Therefore the spider was {\em not} included in the derivation of stability requirements.

Studies were carried out with both monochromatic and polychromatic PSFs. From the perspective of stability requirements, the most important issue is that while the ellipticity of a {\em monochromatic} PSF is {\em quadratic} in the wavefront error \citep[e.g.][]{2010SPIE.7731E..37N}, the ellipticity of a {\em polychromatic} PSF can contain linear terms in the wavefront error if there is lateral color, as occurs in WFIRST since the beam intersects the filter at a nonzero angle of incidence.\footnote{In CCD systems lateral color can be produced in the detector due to the large mean free path of red photons in silicon. For WFIRST, using thin direct-bandgap detectors, this effect is much smaller than the lateral color induced by the optics.} Thus for small wavefront errors it is possible that a wavefront stability budget based on the quadratic dependences of PSF ellipticity (e.g. defocus$\times$astigmatism) and monochromatic PSFs may be too optimistic. The lateral color mixes with odd aberrations to produce PSF ellipticity, so a special ``coma$\times$lateral color'' contribution is included.

\cmnt{Put this where? Furthermore, additive and multiplicative biases can also be introduced by imperfect shape measurement algorithms even in the presence of perfect knowledge of the PSF. The early shape measurement algorithms suffered severe difficulties in both additive and multiplicative biases \citep[e.g.][]{2001MNRAS.325.1065B, 2001A&A...366..717E, 2003MNRAS.343..459H}, but much progress has been made recently in both mock data challenges \citep{2006MNRAS.368.1323H, 2007MNRAS.376...13M, 2010MNRAS.405.2044B} and in the development of Fourier-space shape measurement techniques that are ``exact'' for isolated galaxies in the presence of perfect PSF knowledge, even given complicated morphology \citep{2010MNRAS.406.2793B}.}

